\chapter{Objectives}
\section{Internship subject}
My official internship subject was : "Research about noise and reverberation reduction for robust speech recognition".
However, it was only there both as a starting point and a lead: the development of my subject took place during the internship.
There were two major distinct parts to it :
\begin{itemize}
\item An engineering part, where I had to develop a software library for noise reduction and have speech recognition running efficiently on an embedded device.
\item A research part, where I had to come up with novel ideas to improve on the current methods.
\end{itemize}


\section{Engineering objectives}
These are the objectives I had to achieve from an engineering point of view.
\subsection{Working with embedded systems}
The final goal for my work was to have the speech recognition working on an embedded device, with added noise reduction techniques.
Depending on the device, this could lead to constraints for the system.
I was given a choice between different families of devices : 
\begin{itemize}
\item A \ac{FPGA}
\item An \brand{Arduino} board
\item A \brand{BeagleBoard}
\item A \ac{DSP}
\end{itemize}
I rejected the idea of the \ac{FPGA} and the \ac{DSP} because I would have had to write \ac{VHDL} code, which I don't know, and because speech recognition also has a lot of high-level logic, like grammar, which would be really difficult to implement with \ac{DSP} patterns, which are better suited for simple and repetitive tasks.
I didn't have knowledge of \brand{Arduino}, so I choose to work with the \brand{BeagleBoard}, because it runs Linux natively, which makes development very easy and allows me to focus only on the important matters.
\subsubsection{Presentation of the \brand{BeagleBoard-XM}}
The model on which I had to work was the \brand{BeagleBoard-XM}. It is originally a \brand{Texas Instruments} test board, which acts like a full-fledged computer. Some of its relevant characteristics are : 
\begin{itemize}
\item 1 GHz ARM Cortex-A8 processor. 512 MB of RAM.
\item USB / HDMI / Ethernet connectivity
\item No embedded memory : the operating system is on a micro-SD card.
\end{itemize}

It ships with \brand{Angstrom Linux}, but packages are not always up to date, and I have a better knowledge of \brand{Debian GNU/Linux}, so I decided to switch to it.

\subsection{Speech recognition}
There are many software able to perform a fairly efficient speech recognition, so I had to choose one. I would have to make this software work on the \brand{BeagleBoard}. I would also have to include the noise reduction algorithms inside the software, so it had to be modular.
\subsubsection{A Japanese software : \brand{Julius}}
 \brand{Julius} is a speech recognition software developed at \ac{NITECH}. It is very mature : development started in 1991 and the latest version came out during my internship.
 
\subsubsection{Acoustic and Language models}
To perform speech recognition, the software needs some data, which is generally separated 
in : 
\begin{enumerate}
\item A purely audio-data related model : the acoustic model. It generally uses \acp{HMM}.
\item A semantic-data related model : the language model. It is made from a dictionary and a grammar. The grammar can be made by hand for little corpuses, like a phone dialer, but for a full human language it requires the use of probabilistic word position estimation known as \textit{n-grams}.
\end{enumerate}

During the course of my internship, I had to build both a language and acoustic model for use with \brand{Julius}.

\subsubsection{\ac{HTK}}
\ac{HTK} is a set of software currently maintained by the \ac{CUED}, which is used to build acoustic and language models. I would have to learn how to use them by reading the HTK Book, the reference book on the subject.

\subsection{Noise reduction}
The core of my internship was about performing noise reduction on a real-time speech signal.
I would first have to read scientific literature on the subject, and would then move to implementation.
Simply put, noise reduction is generally a two step process, which often occurs in the spectral domain : 
\begin{enumerate}
\item Noise estimation.
\item Noise removal according to the estimation.
\end{enumerate}

Each of these steps are active research fields, so there are many different algorithms for each. I was given some of these papers during the whole course of my internship, which led to important architectural changes in the end. I will give a brief explanation of all the papers I was given.
\subsubsection{Spectral subtraction}
Spectral subtraction is the \textit{de facto} standard method for noise removal. The concept is simple : the estimated noise spectrum is removed to the signal spectrum for every frame of signal.

\paragraph{Standard algorithm}
The original method was described by Steven Boll in 1979\cite{boll1979suppression}. A standard optimization is to iterate the process many times, it is then called Iterative Spectral Subtraction. The first objective of my internship was to implement this algorithm.
\paragraph{Equal-Loudness optimization}
Another optimization based on psychoacoustic properties was developed in the lab\cite{horii2013musical}. It filters each band of the noise spectrum using the equal-loudness contour curves, which represent how the human ear filters the signal.
\paragraph{Geometric algorithm}
This algorithm\cite{lu2008geometric} differs of standard spectral subtraction by taking into account otherwise neglected coefficients. It offers a high level of performance.
\subsubsection{Noise estimation}
As for spectral subtraction, there are multiple ways to estimate the ambient noise present on a signal.
\paragraph{Simple method}
A naive method is to assume that the spectrum of the noise will be constant, and that the signal begins with a speechless part. While this estimation method is inefficient in a changing environment, it works reasonably well if the noise is effectively constant, like with white noise.
\paragraph{Method based on statistics}
This method is presented in a paper from Rainer Martin\cite{martin2001noise}. In opposition to the previous one, it is constantly running, even on frames containing speech, and offers a good noise estimation even if the noise changes completely. It is based on a statistical study of the signal spectrum.
\section{Research objectives}
In addition to the practical implementation of common algorithms, I had to carry some research work.
\subsection{Improving spectral subtraction}
The main research goal was to improve the spectral subtraction method : I would have to come up with a novel idea, and implement, test, and compare it.
\subsection{Evaluating the different methods}
In parallel, I would have to conduct global evaluation of all the methods proposed beforehand.
\section{Resources offered by the lab}
\begin{itemize}
\item Access to scientific papers from \ac{IEEE}, Elsevier and other publishers via a school-wide subscription.
\item A \brand{BeagleBoard-XM}.
\item Audio equipment to perform experimentation: microphone, loudspeaker, headphone, microphone preamplifier. 
\item Voice database : \ac{TIMIT} (english), \ac{JNAS} (japanese).
\item Noise database : \ac{JEIDA}.
\item A computer screen.
\item I was proposed a laptop but the keyboard and operating system (\brand{Microsoft\textregistered Windows\texttrademark}) were in Japanese so I preferred to use mine.

\end{itemize}
\section{Planning}
Since my objectives evolved a lot with time, I will rather present how I organized myself from week to week.
In the lab, I had to do a weekly report to Mr. Nishiura, and make a short powerpoint presentation explaining what I was doing, why, and what I would be doing next.

\begin{itemize}
\item The first and second week were about learning how to use the standard tools, algorithms, etc. For instance, how did the Fast Fourier Transform work, how to make a simple digital filter, the overlap-add and windowing techniques, etc. I read many papers during this period, as well as some parts of the book \textit{The Scientist and Engineer's Guide to Digital Signal Processing}. The libraries I learned were : 
\begin{itemize}
\item \ac{FFTW}
\item \ac{ALSA}
\item The audio-related parts of \textbf{Qt}, especially the \textbf{QtMultimedia} module which enacts low-level audio control.
\end{itemize}
\item On the third week, I started prototyping some simple \textbf{C} functions to perform the basic spectral subtraction algorithm. I also started coding a \textbf{Qt} windowed application to test the algorithms on audio files. Finally, I worked on a script to generate a suitable acoustic and language model from the \ac{TIMIT} corpus.
\item On the fourth week, I started to learn how to use \brand{Julius}, the speech synthesis engine, and its library \brand{LibJulius}, as well as \ac{HTK}. In parallel, I would start implementing my research ideas, and looking for an efficient \ac{CWT} library. I would also add some batch processing abilities to my test application.
\item On the fifth week, I started preparing a poster about my research that I would present at the Kansai Joint Seminar. Also, I finished coding most of the algorithms in the papers that were cited beforehand. At the same time, I tried to learn how to use most of the \brand{C++} \brand{STL}, as well as \brand{C++11} features.
\item The sixth and seventh week were mostly about preparing the Kansai Joint Speech Seminar : learning how to make a good poster, checking and trying to improve the algorithms, making measurements and finally presenting it.
\item On the seventh week I did a huge refactoring to make the library more easily expandable.
\item On the eight week, I would port all the software on the \brand{BeagleBoard} and check that everything would work correctly. There was also a lot of configuration to do to manage the audio input correctly.
\item On the ninth week, I did testing and comparing the software on the beagleboard with sound databases and real speakers, as well as noises emitted with a loudspeaker, to simulate performance in a real noisy environment
\item The tenth week was a national holiday week in Japan : I focused on the internship report, and did tests with a Japanese corpus.
\item The two last weeks were mostly about bug-fixing and preparing my final presentation.
\end{itemize}
